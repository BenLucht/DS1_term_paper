\subsection{Scikit-Learn MeanShift}
\label{subsec:sklearn_meanshift}
\textit{written by L.B.}\\

For Mean Shift Clustering we use the implementation provided from \cite{sklearn_api} in version 0.24.2 which is the latest stable release as of \today. 
The implementation offers the possibility to enter several additional parameters supplementary to the required bandwidth. Three related additional parameters can be used to speed up the clustering procedure as mentioned in section \ref{sec:Mean Shift}. These parameters are called \mintinline[bgcolor=code-bg]{python}{seeds}, \mintinline[bgcolor=code-bg]{python}{bin_seeding} and \mintinline[bgcolor=code-bg]{python}{min_bin_freq} and allow to define which of the given data points should be used for the initialization of candidate centroids. Exclusively if no array of initial seeds is passed over, the algorithm checks the \mintinline[bgcolor=code-bg]{python}{bin_seeding} parameter value. If this parameter is \mintinline[bgcolor=code-bg]{python}{False}, all given data points are used to initialize kernels. In contrast to that, if the \mintinline[bgcolor=code-bg]{python}{bin_seeding} parameter is \mintinline[bgcolor=code-bg]{python}{True} , the algorithm creates a grid with cell size defined by the bandwidth parameter. The number of points within the created bins defines which data points are used as seeds. Therefore, \mintinline[bgcolor=code-bg]{python}{min_bin_freq} defines the minimum number of data points that have to occur in a bin to define a candidate centroid.
Running the algorithm in its default mode, as it is done in this work, all the given data points are initialized as candidate centroids and hence used as initial kernel locations. 
In addition to that, the algorithm provides a parameter called \mintinline[bgcolor=code-bg]{python}{cluster_all}, which defines whether all the given data points have to be assigned to a cluster. If the parameter is \mintinline[bgcolor=code-bg]{python}{True}, all data points are clustered, even if they are not placed within any kernel. Those outliers are assigned to their nearest kernel. If the parameter is \mintinline[bgcolor=code-bg]{python}{False}, outliers outside any kernel's radius are not assigned to any cluster, but instead their label is \mintinline[bgcolor=code-bg]{python}{-1}. 
A technical parameter is called \mintinline[bgcolor=code-bg]{python}{n_jobs}. Users can pass an integer to determine the number of jobs which should be used for the computation.\\
In theory, the algorithm stops the mean shift iteration for each seed if it converges. The \mintinline[bgcolor=code-bg]{python}{max_iter}parameter can be used to fix a maximum number of iterations if a seed does not converge. In the default mode, the maximum number of iterations is set to 300.\\
If no value for the required bandwidth parameter is provided, the bandwidth is estimated using a built-in \mintinline[bgcolor=code-bg]{python}{estimate_bandwidth} function. This function calculates the median of all pairwise distances and ascertains a suitable bandwidth based on the underlying dataset.\newline
The algorithm performs the mean shift procedure for all the seeds in parallel. In each iteration, all the neighbors within the kernel's radius are found by executing a nearest neighbor search. The seed is shifted to the mean of all the neighboring data points until it either converges or the maximum number of iterations is reached. 
After shifting seeds, some postprocessing steps are performed. The algorithm searches for seeds whose distance is smaller than the bandwidth. If duplicates like these are found, the one with fewer data points is removed. If \mintinline[bgcolor=code-bg]{python}{cluster_all} is \mintinline[bgcolor=code-bg]{python}{True}, each of the given data points is assigned to the cluster defined by the seed that is closest to the respective data point. If \mintinline[bgcolor=code-bg]{python}{cluster_all} is \mintinline[bgcolor=code-bg]{python}{False}, only those data points that are located within a kernel's bandwidth are assigned to the cluster defined by the nearest seed and the remaining data points are assigned to the artificial cluster with label \mintinline[bgcolor=code-bg]{python}{-1}.