\subsection{Scikit-Learn Spectral Clustering}
\label{subsec:sklearn_spectral}

For Spectral Clustering we use the implementation provided from [26] in version 0.24.2 which is the latest stable release as of June 20, 2021. The implementation offers the possibility to enter several parameters, some of them are necessary to set, where for some it is fair enough to use the default values.\newline
The \verb|n_clusters| show the dimension of the projection subspace and take an Integer as an Input value. The default value is 8. In our program we used to equal \verb|n_clusters| to k, so that it can be set flexibly as a parameter. \newline
The affinity matrix describes the relationship of the samples to embed. Possible examples are the k-nearest neighbours connectivity of the samples, the heat kernel of the pairwise distance matrix of the samples or the symmetric k-nearest neighbours connectivity matrix of the samples, which have already been described in 2.4. \newline
According to this, the construction of the affinity matrix can be computed by setting it equal to {\verb|nearest_neigbours|}, as it is done in our program. To set the affinity matrix one can also use {\verb|precomputed|} which interprets X as a precomputed affinity matrix, where larger values indicate greater similarity between instances. \newline
Also {\verb|precomputed_nearest_neighbours|} can be set equal to affinity, so that X is interpreted as a sparse graph of precomputed distances. Consequently, it constructs a binary affinity matrix from \verb|the n_neighbours| nearest neighbours of each instance. \newline
In addition to that, the algorithm provides a parameter called \verb|n_init|, which sets the number of times the k-means algorithm will run with different centroid data points.  Running the algorithm in its default mode, it gets the value 10. In our work, we used 100 as the value for \verb|n_init|.
As mentioned before the affinity matrix is used for clustering, but is only available after calling the method fit, which performs spectral clustering from features or affinity matrix. If one wants to get cluster labels returned, the method \verb|fit_predict| must be used. This can also be seen in our program. \newline
An additional parameter which can be used for Spectral clustering is the eigenvalue decomposition strategy \verb|eigen_solver|. The default value takes \textit{arpack} and if \textit{amg} should be used, then its necessary to install pyamg. The number of eigenvectors to use for the spectral embedding is named as the parameter \verb|n_components| and takes an integer as an input. Its default value is the \verb|n_clusters|. It was not being set in our program intentionally.
Another interesting parameter named \verb|gamma| represents the kernel coefficient for rbf( constructs the affinity matrix using a radial basis function), Laplacian and other kernels. But in case of setting the affinity parameter to {\verb|nearest_neigbours|} , the gamma parameter is being ignored, which is the case in our program, too. \newline
The parameter \verb|n_neighbours| is being used when the affinity matrix is being constructed using the nearest neighbours method. For affinity = \textit{rbf} the parameter is being ignored and if no value for the \verb|n_neighbours| parameter is provided, the default value 10 is taken. \newline
There are two ways of assigning labels after the Laplacian embedding. Mostly, \textit{kmeans} is used for the parameter \verb|assign_labels|, that is why it is also the default value of the parameter. Nevertheless, there is another option using \verb|assign_labels| as \textit{discretize}, which is less sensitive when initialized randomly than kmeans. The parameter \verb|n_jobs| decides the number of parallel jobs to run when affinity = \verb|nearest_neighbours| or \verb|precomputed_nearest_neighbours|. None is set as the default value. It means in general 1 unless we are in a \verb|joblib.parallel_backend| context and -1  means using all processors. \newline