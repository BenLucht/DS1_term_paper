\subsection{Spectral Clustering}
\textit{written by M.A.}\\

Spectral clustering is an unsupervised learning algorithm. It treats each data point as a graph-node and performs the clustering problem into a graph-partitioning problem. The affinity rather then the absolute location (i.e. k-means) determines what points are set in which cluster.
The algorithm consists of four basic steps: \newline

1.	The construction of a similarity graph:\newline
During this step the Similarity Graph is being built in the form of an adjacent matrix, which is presented by A. The adjacency matrix can be built as an Epsilon-neighbourhood Graph, K-Nearest Neighbours strategy or a fully-connected graph. \newline

The Epsilon-neighbourhood Graph sets the parameter epsilon already. Afterwards each point is connected to all the points which lie in the epsilon -radius. The graph which is built in this case is an undirected and unweighted graph. \newline

By using the k-nearest neighbours strategy a parameter k is set as well. Then, for two vertices u and v, an edge is directed from u to v only if v is among the k-nearest neighbours of u. 
To build the fully-connected graph, each point is connected with an undirected edge-weighted by the distance between the two points to every other point. Since this approach is used to model the local neighbourhood relationships thus typically the Gaussian similarity metric is used to calculate the distance. \newline


2.	Projecting the data onto a lower Dimensional Space: This step is done to account for the possibility that members of the same cluster may be far away in the given dimensional space. Thus the dimensional space is reduced so that those points are closer in the reduced dimensional space and thus can be clustered together by a traditional clustering algorithm. It is done by computing the Graph Laplacian Matrix . To compute it though first, the degree of a node needs to be defined. The degree of the $i$th node is given by (Formel noch einfügen) \newline

Note that $w_{i,j}$, is the edge between the nodes i and j as defined in the adjacency matrix above. The degree matrix is defined as follows (noch einfügen) \newline

This Matrix is then normalized for mathematical efficiency. To reduce the dimensions, first, the eigenvalues and the respective eigenvectors are calculated. If the number of clusters is k then the first eigenvalues and their eigenvectors are taken and stacked into a matrix such that the eigenvectors are the columns. \newline

3.	Clustering the Data: This process mainly involves clustering the reduced data by using any traditional clustering technique – typically K-Means Clustering. First, each node is assigned a row of the normalized of the Graph Laplacian Matrix. Then this data is clustered using any traditional technique. To transform the clustering result, the node identifier is retained. \newline